\documentclass[a4paper, 11pt]{scrartcl}

% ----- Packages -----
\usepackage{xspace}
\usepackage{scrlayer-scrpage}                           % header and footer
\usepackage{microtype}
\usepackage[unicode,colorlinks=false,breaklinks=true]{hyperref} % make e.g. toc clickable
\usepackage[english=american]{csquotes}                  % pick always the right quotation marks
% TODO remove this eventually before merging
\usepackage{todo}                                       % to enter todos in the document
\usepackage{pdflscape}
\usepackage{cleveref}

\usepackage{tabularx}         % better configuration of tables
\usepackage{ltxtable}         % use tabularx as longtable (\LTXtable{width}{pathToTable}) where the table referenced to contains of \begin{longtable}{preamble}content\end{longtable}
\usepackage{booktabs}         % more nice rules (headrule/midrule/crule/bottomrule)

\usepackage[svgnames]{xcolor}
\usepackage{tcolorbox}
\tcbuselibrary{skins,listingsutf8,documentation}
% TODO adjust colors?
\renewcommand*{\tcbdocnew}[1]{\textcolor{green!50!black}{\sffamily\bfseries N} #1}
\renewcommand*{\tcbdocupdated}[1]{\textcolor{blue!75!black}{\sffamily\bfseries U} #1}
\tcbset{skin=enhanced,
	doc head={colback=yellow!10!white,interior style=fill},
	doc head key={colback=magenta!5!white,interior style=fill},
	doc head path={colback=blue!50!gray!7!white,interior style=fill},
	color key=DarkViolet,
	color value=Teal,
	color color=Teal,
	color counter=Orange!85!black,
	color length=Orange!85!black,
	index colorize,
	index annotate,
}

\usepackage{multicol}
\usepackage{catppuccinpalette}
\usepackage{listings} % for code

\lstset{
	rulecolor=\color{black},
	emphstyle=\color{CtpYellow},
	basicstyle=\tiny\ttfamily,
	stringstyle=\color{CtpGreen},
	commentstyle=\color{CtpOverlay2},
	keywordstyle=\color{CtpMauve},
	backgroundcolor=\color{CtpBase},
	frame=none,
	%
	columns=fullflexible, % columns of chars
	tabsize=2,
	%
	captionpos=b,
	%
	breakautoindent=true, %
	postbreak=\mbox{\textcolor{red}{$\hookrightarrow$}\space},
	breaklines=true,
	%
	showstringspaces=false,
	showtabs=false,
	showspaces=false,
	keepspaces=true,
	%
	inputencoding=utf8, extendedchars=true,
	literate=
	{á}{{\'a}}1 {é}{{\'e}}1 {í}{{\'i}}1 {ó}{{\'o}}1 {ú}{{\'u}}1
	{Á}{{\'A}}1 {É}{{\'E}}1 {Í}{{\'I}}1 {Ó}{{\'O}}1 {Ú}{{\'U}}1
	{à}{{\`a}}1 {è}{{\`e}}1 {ì}{{\`i}}1 {ò}{{\`o}}1 {ù}{{\`u}}1
	{À}{{\`A}}1 {È}{{\'E}}1 {Ì}{{\`I}}1 {Ò}{{\`O}}1 {Ù}{{\`U}}1
	{ä}{{\"a}}1 {ë}{{\"e}}1 {ï}{{\"i}}1 {ö}{{\"o}}1 {ü}{{\"u}}1
	{Ä}{{\"A}}1 {Ë}{{\"E}}1 {Ï}{{\"I}}1 {Ö}{{\"O}}1 {Ü}{{\"U}}1
	{â}{{\^a}}1 {ê}{{\^e}}1 {î}{{\^i}}1 {ô}{{\^o}}1 {û}{{\^u}}1
	{Â}{{\^A}}1 {Ê}{{\^E}}1 {Î}{{\^I}}1 {Ô}{{\^O}}1 {Û}{{\^U}}1
	{Ã}{{\~A}}1 {ã}{{\~a}}1 {Õ}{{\~O}}1 {õ}{{\~o}}1
	{œ}{{\oe}}1 {Œ}{{\OE}}1 {æ}{{\ae}}1 {Æ}{{\AE}}1 {ß}{{\ss}}1
	{ű}{{\H{u}}}1 {Ű}{{\H{U}}}1 {ő}{{\H{o}}}1 {Ő}{{\H{O}}}1
	{ç}{{\c c}}1 {Ç}{{\c C}}1 {ø}{{\o}}1 {å}{{\r a}}1 {Å}{{\r A}}1
	{€}{{\euro}}1 {£}{{\pounds}}1 {«}{{\guillemotleft}}1
	{»}{{\guillemotright}}1 {ñ}{{\~n}}1 {Ñ}{{\~N}}1 {¿}{{?`}}1
}

% ----- Settings -----
\KOMAoptions{
	paper=a4,           % set papersize
	fontsize=12pt,      % set fontsize to 12pt
	parskip=half,       % half a line will seperate paragraphs
	headings=normal,    % headings in normalsize
	BCOR=0cm,           % no space for binding
	DIV=calc,           % calculare margins automatically, the calculated values can be found in the log file or use \areaset[BCOR]{width}{height}} to explicitly specify the textarea
	twoside=false,      % onesided document
	twocolumn=false,    % one columned pages
	draft=false,         % show additional information
	numbers=autoendperiod, % automatic punctuation in outline
}
\parindent0pt

% ----- Commands -----
\newcommand\eg{e.\,g.\xspace}
\newcommand\CluttealTeX{ClutTeal\TeX\xspace}
\providecommand\BibTeX{\textsc{Bib}\TeX\xspace}

\newtcolorbox{boxcmd}{fontupper=\ttfamily}

\NewTCBListing{boxcodelua}{}{%
	listing options={style=tcblatex,language={[5.3]Lua},basicstyle=\ttfamily\scriptsize},
	listing only,
}

\newcommand\texcmd[1]{\texttt{\textbackslash #1}}
\newcommand\texenv[1]{\texttt{#1}}
\newcommand\texpkg[1]{\texttt{#1}}
\newcommand\metavar[1]{\textnormal{\textsf{#1}}}

\title{\CluttealTeX manual\\(Version 0.8.0)} % TODO(release) update
\author{%
	Lukas Heindl\thanks{thanks to ARATA Mizuki for writing cluttex of which \CluttealTeX is a fork}\\
	\url{oss.heindl+latex@protonmail.com} \\
	\url{https://github.com/atticus-sullivan/cluttealtex}
}
\date{2024-03-11} % TODO(release) update

\begin{document}
\maketitle
\tableofcontents

\section{About \CluttealTeX}
\CluttealTeX is an automation tool for \LaTeX\ document processing.
Basic features are,
\begin{itemize}
	\item Does not clutter your working directory with \enquote{extra} files, like \texttt{.aux} or \texttt{.log}.
	\item If multiple runs are required to generate correct document, do so.
	\item Watch input files, and re-process documents if changes are detected\footnote{needs an external program if you are on a Unix system}.
	\item Run MakeIndex, \BibTeX, Biber, and some other tools if requested and necessary.
	\item Produces a PDF, even if the engine (e.g.\ p\TeX) does not suport direct PDF generation.
		If you want a DVI file, use \texttt{--output-format=dvi} option.
	\item No external dependencies, \CluttealTeX only uses \texttt{texlua} which should come with TeXLive
\end{itemize}

The unique feature of this program is that, auxiliary files such as \texttt{.aux} or \texttt{.toc} are created in an isolated location, so you will not be annoyed with these extra files.

Other similar programs include
\begin{itemize}
	\item \href{https://ctan.org/pkg/latexmk/}{latexmk}
	\item \href{https://ctan.org/pkg/arara/}{arara}
	\item \href{https://gitlab.com/latex-rubber/rubber/}{rubber}
\end{itemize}

\subsection{Feedback}
If any issues arise, be it a bug, a missing feature and\,/\,or missing support for an external tool\,/\,package, feel free to reach out and file an issue at \url{https://github.com/atticus-sullivan/cluttealtex/issues}.

\section{How to use \CluttealTeX}
\subsection{Installation}
Installing manually:
\begin{enumerate}
	\renewcommand{\theenumi}{\arabic{enumi}}
	\item fetch an archive from GitHub\footnote{\url{https://github.com/atticus-sullivan/cluttealtex}}
		(either use the latest release for the stable build or from latest github actions for the nightly build)
	\item extract the archive (or directly only download the executable)
	\item copy \texttt{bin/cluttealtex} or \texttt{bin/cluttealtex.bat} to somewhere in your \texttt{PATH}
\end{enumerate}

Installing via CTAN (\eg \texttt{tlmgr}) currently is not possible, but uploading this as CTAN package is planned in the near future.

\subsubsection{Building}
You can also build \CluttealTeX yourself:
\begin{enumerate}
	\renewcommand{\theenumi}{\arabic{enumi}}
	\item Clone the repository from GitHub
	\item run \texttt{make install}. For this you need the tools
		\begin{itemize}
			\item \texttt{fd}
			\item \texttt{tl} (installable via \texttt{luarocks})
			\item \texttt{l3build}
		\end{itemize}
		this will install \CluttealTeX into your \texttt{TEXMFHOME} tree
\end{enumerate}

\subsection{Quickstart}
For building a simple (pdf)\LaTeX document running:
\begin{boxcmd}
  cluttealtex main.tex
\end{boxcmd}

If you need a different \TeX engine like Lua\LaTeX or Xe\LaTeX, you can specify this as well:
\begin{boxcmd}
  cluttealtex -e ENGINE main.tex
\end{boxcmd}

Similarly if your main file is not called \texttt{main.tex} you only have to adjust this parameter:
\begin{boxcmd}
  cluttealtex -e ENGINE MAINFILE.tex
\end{boxcmd}

\subsection{Usage}
The general usage is as follows:
\begin{boxcmd}
  cluttealtex -e ENGINE OPTIONs [--] INPUT.tex
\end{boxcmd}

Also \CluttealTeX will search your current working directory for a file called \texttt{.cluttealtexrc.lua}.
If it finds such a file, it will execute it and use the returned table for the options as well.
This way, you can achieve it that you can build your document by only executing
\begin{boxcmd}
	cluttealtex
\end{boxcmd}
without any options.

The config file can contain all kinds of tables.
If one is missing \CluttealTeX will use an empty table (having no effect) instead.
Also note that arguments passed via the CLI always have a higher priority than arguments passed via the config file.
In the future some options might only be available via the config file as setting them via CLI is too complicated\,/\,not feasible (like \eg adding a hook which gets called after every successful compilation).

A fully populated config file looks as follows:
\begin{boxcodelua}
return {
	options = {
		file = "main.tex",
		output_directory = "tex-aux",
		change_directory = true,
		engine = "pdflatex",
		biber = true, -- uses the default value of biber
		-- could also be an array of these tables to configure multiple glossaries
		-- equivalent to "--glossaries=makeindex:acr:acn:alg"
		glossaries = {
			type="makeindex",
			out="acr",
			inp="acn",
			log="alg"
		},
		-- just something high to disable this upper limit
		max_iterations = "50",
		quiet = "0",
	},
	-- Change/Set the default value of an option. Only has an effect on cli arguments.
	defaults = {
		-- now if you pass --watch, "inofity" will be passed by default instead of "auto"
		watch = "inotify",
	},
	-- for the experts:
	add_cli_options = {
		-- adds a new cli argument which can set arbitrary other options
		my_own_option = {
			-- short = "m",
			long = "my-own-option",
			param = true,
			default = "my default value",
			boolean = false,
			-- passing this multiple times accumulates the values
			-- (only has a readl effect when reading options from config file)
			accumulate = true,
			handle_cli = function(options, param)
				-- my code, checking the passed parameter and setting some options accordingly
				-- of course here you need to be careful not to leave "options" in an invalid state
			end,
		}
	}
}
\end{boxcodelua}

Generally in the config, but particularly for the passed functions, you can write arbitrary Lua code.
The environment of the config and the environment of \CluttealTeX is partially shared.
When the config accesses a global variable, it gets passed a (deep) copy of that value to avoid the config intervening with the execution of \CluttealTeX.

You find a complete list of all available options at the end of this documentation in \cref{sec:listOpt}.
The following sections will explain the various options in detail, grouped by category.

\subsubsection{Basic Options}

% \begin{docKeys}[
%		% doc parameter={=\meta{engine}},
%]{
% 		{
% 			% doc new=2025-01-01,
% 			% doc updated=2025-01-01,
%			doc keypath=cli,
% 			doc name=suggest,
% 			doc parameter={=\meta{value}},
% 			doc description=initial default,
% 		},
% 		{
% 			% doc new=2025-01-01,
% 			% doc updated=2025-01-01,
%			doc keypath=config,
% 			doc name=suggest,
% 			doc parameter={=\meta{value}},
% 			doc description=initial default,
% 		},
% 	}
% 	description
% \end{docKeys}
% \tcbdocmarginnote{\tcbdocnew{2025-01-01}}
% \tcbdocmarginnote{\tcbdocupdated{2025-01-01}}
% \refKey{suggestCLI}

\paragraph{\TeX}
\begin{docKeys}[
		doc parameter={=\meta{engine}},
	]{
		{
			% doc new=2025-01-01,
			% doc updated=2025-01-01,
			doc keypath=cli,
			doc name=engine,
			doc description={default: none, initial: none, \textcolor{CtpRed}{\bfseries\sffamily required}},
		},
		{
			% doc new=2025-01-01,
			% doc updated=2025-01-01,
			doc keypath=config,
			doc name=engine,
			doc description={default: none, initial: none, \textcolor{CtpRed}{\bfseries\sffamily required}},
		},
	}
	Set which \TeX\ engine/format to use.
	\metavar{ENGINE} is one of the following:
	\begin{multicols}{3}
		\begin{itemize}
			\item \texttt{pdflatex}
			\item \texttt{pdftex}
			\item \texttt{lualatex}
			\item \texttt{luatex}
			\item \texttt{luajittex}
			\item \texttt{xelatex}
			\item \texttt{xetex}
			\item \texttt{latex}
			\item \texttt{etex}
			\item \texttt{tex}
			\item \texttt{platex}
			\item \texttt{eptex}
			\item \texttt{ptex}
			\item \texttt{uplatex}
			\item \texttt{euptex}
			\item \texttt{uptex}
			\item[]
			\item[]
		\end{itemize}
	\end{multicols}
\end{docKeys}
\begin{docKeys}[
		doc parameter={=\meta{engine\_executable}},
	]{
		{
			% doc new=2025-01-01,
			% doc updated=2025-01-01,
			doc keypath=cli,
			doc name=engine-executable,
			doc description={default: none, initial: none},
		},
		{
			% doc new=2025-01-01,
			% doc updated=2025-01-01,
			doc keypath=config,
			doc name=engine_executable,
			doc description={default: none, initial: none},
		},
	}
	\Todo{what for?}
\end{docKeys}
\begin{docKeys}{
		{
			% doc new=2025-01-01,
			% doc updated=2025-01-01,
			doc keypath=cli,
			doc name=tex-option,
			doc description={default: none, initial: none},
			doc parameter={=\meta{tex\_option}},
		},
		{
			% doc new=2025-01-01,
			% doc updated=2025-01-01,
			doc keypath=cli,
			doc name=tex-options,
			doc description={default: none, initial: none},
			doc parameter={=\meta{tex\_options}},
		},
		{
			% doc new=2025-01-01,
			% doc updated=2025-01-01,
			doc keypath=config,
			doc name=tex_option,
			doc description={default: none, initial: none},
			doc parameter={=\meta{tex\_option}},
		},
		{
			% doc new=2025-01-01,
			% doc updated=2025-01-01,
			doc keypath=config,
			doc name=tex_options,
			doc description={default: none, initial: none},
			doc parameter={=\meta{tex\_options}},
		},
	}
	Pass extra options to \TeX.
\end{docKeys}
\begin{docKeys}{
		{
			% doc new=2025-01-01,
			% doc updated=2025-01-01,
			doc keypath=cli,
			doc name=dvipdfmx-option,
			doc description={default: none, initial: none},
			doc parameter={=\meta{dvipdfmx\_option}},
		},
		{
			% doc new=2025-01-01,
			% doc updated=2025-01-01,
			doc keypath=cli,
			doc name=dvipdfmx-options,
			doc description={default: none, initial: none},
			doc parameter={=\meta{dvipdfmx\_options}},
		},
		{
			% doc new=2025-01-01,
			% doc updated=2025-01-01,
			doc keypath=config,
			doc name=dvipdfmx_option,
			doc description={default: none, initial: none},
			doc parameter={=\meta{dvipdfmx\_option}},
		},
		{
			% doc new=2025-01-01,
			% doc updated=2025-01-01,
			doc keypath=config,
			doc name=dvipdfmx_options,
			doc description={default: none, initial: none},
			doc parameter={=\meta{dvipdfmx\_options}},
		},
	}
	Pass extra options to \texttt{dvipdfmx}.
\end{docKeys}
\begin{docKeys}[
		doc parameter={=\meta{interaction}},
	]{
		{
			% doc new=2025-01-01,
			% doc updated=2025-01-01,
			doc keypath=cli,
			doc name=interaction,
			doc description={default: none, initial: none},
		},
		{
			% doc new=2025-01-01,
			% doc updated=2025-01-01,
			doc keypath=config,
			doc name=interaction,
			doc description={default: none, initial: none},
		},
	}
	\meta{interaction} is one of
	\begin{multicols}{3}
		\begin{itemize}
			\item \texttt{batchmode}
			\item \texttt{nonstopmode}
			\item \texttt{scrollmode}
			\item \texttt{errorstopmode}
			\item[]
			\item[]
		\end{itemize}
	\end{multicols}
	% Default: \texttt{nonstopmode}
\end{docKeys}
\begin{docKeys}[
		doc parameter={=\meta{jobname}},
	]{
		{
			% doc new=2025-01-01,
			% doc updated=2025-01-01,
			doc keypath=cli,
			doc name=jobname,
			doc description={default: none, initial: none},
		},
		{
			% doc new=2025-01-01,
			% doc updated=2025-01-01,
			doc keypath=config,
			doc name=jobname,
			doc description={default: none, initial: none},
		},
	}
\end{docKeys}
\begin{docKeys}[
		doc parameter={=\meta{output}},
	]{
		{
			% doc new=2025-01-01,
			% doc updated=2025-01-01,
			doc keypath=cli,
			doc name=output,
			doc description={default: none, initial: none},
		},
		{
			% doc new=2025-01-01,
			% doc updated=2025-01-01,
			doc keypath=config,
			doc name=output,
			doc description={default: none, initial: \texttt{\meta{jobname}.\meta{format}}},
		},
	}
	Set output file name.
\end{docKeys}
\begin{docKeys}[
	]{
		{
			% doc new=2025-01-01,
			% doc updated=2025-01-01,
			doc keypath=cli,
			doc name=shell-escape,
			doc description={enables when passed, initial: \texttt{false}},
		},
		{
			% doc new=2025-01-01,
			% doc updated=2025-01-01,
			doc keypath=cli,
			doc name=no-shell-escape,
			doc description={disables when passed},
		},
		{
			% doc new=2025-01-01,
			% doc updated=2025-01-01,
			doc keypath=config,
			doc name=shell_escape,
			doc description={initial: \texttt{false}},
		},
	}
	\Todo{}
	% This option cannot be combined with \refKey{/config/shell_restricted}/\refKey{/cli/shell-restricted}.
\end{docKeys}
\begin{docKeys}[
	]{
		{
			% doc new=2025-01-01,
			% doc updated=2025-01-01,
			doc keypath=cli,
			doc name=shell-restricted,
			doc description={enables when passed, initial: \texttt{false}},
		},
		{
			% doc new=2025-01-01,
			% doc updated=2025-01-01,
			doc keypath=cli,
			doc name=no-shell-restricted,
			doc description={disables when passed},
		},
		{
			% doc new=2025-01-01,
			% doc updated=2025-01-01,
			doc keypath=config,
			doc name=shell_restricted,
			doc description={initial: \texttt{false}},
		},
	}
	\Todo{}
	% This option cannot be combined with \refKey{/config/shell_escape}/\refKey{/cli/shell-escape}.
\end{docKeys}
\begin{docKeys}[
		doc parameter={=\meta{synctex}},
	]{
		{
			% doc new=2025-01-01,
			% doc updated=2025-01-01,
			doc keypath=cli,
			doc name=synctex,
			doc description={default: none, initial: none},
		},
		{
			% doc new=2025-01-01,
			% doc updated=2025-01-01,
			doc keypath=config,
			doc name=synctex,
			doc description={default: none, initial: none},
		},
	}
	Generate Sync\TeX\ file.
	Note that \texttt{.synctex.gz} is created alongside the final \texttt{.pdf}.
	\Todo{See \autoref{sec:synctex} for details -- reall in new section?}.
\end{docKeys}
\begin{docKeys}[
	]{
		{
			% doc new=2025-01-01,
			% doc updated=2025-01-01,
			doc keypath=cli,
			doc name=halt-on-error,
			doc description={enables when passed, initial: \texttt{false}},
		},
		{
			% doc new=2025-01-01,
			% doc updated=2025-01-01,
			doc keypath=cli,
			doc name=no-halt-on-error,
			doc description={disables when passed, initial: \texttt{false}},
		},
		{
			% doc new=2025-01-01,
			% doc updated=2025-01-01,
			doc keypath=config,
			doc name=halt_on_error,
			doc description={initial: \texttt{false}},
		},
	}
\end{docKeys}
\begin{docKeys}[
		doc parameter={=\meta{fmt}},
	]{
		{
			% doc new=2025-01-01,
			% doc updated=2025-01-01,
			doc keypath=cli,
			doc name=fmt,
			doc description={default: none, initial: none},
		},
		{
			% doc new=2025-01-01,
			% doc updated=2025-01-01,
			doc keypath=config,
			doc name=fmt,
			doc description={default: none, initial: none},
		},
	}
\end{docKeys}
\begin{docKeys}[
	]{
		{
			% doc new=2025-01-01,
			% doc updated=2025-01-01,
			doc keypath=cli,
			doc name=file-line-error,
			doc description={enables when passed, initial: \texttt{false}},
		},
		{
			% doc new=2025-01-01,
			% doc updated=2025-01-01,
			doc keypath=cli,
			doc name=no-file-line-error,
			doc description={disables when passed, initial: \texttt{false}},
		},
		{
			% doc new=2025-01-01,
			% doc updated=2025-01-01,
			doc keypath=config,
			doc name=file_line_error,
			doc description={initial: \texttt{false}},
		},
	}
\end{docKeys}

\paragraph{\CluttealTeX}
\begin{docKeys}[
	]{
		{
			% doc new=2025-01-01,
			% doc updated=2025-01-01,
			doc keypath=cli,
			doc name=version,
			doc description={short: v},
		},
		{
			% doc new=2025-01-01,
			% doc updated=2025-01-01,
			doc keypath=config,
			doc name=version,
			doc description={},
		},
	}
\end{docKeys}
\begin{docKeys}[
		doc parameter={=\meta{output\_format}},
	]{
		{
			% doc new=2025-01-01,
			% doc updated=2025-01-01,
			doc keypath=cli,
			doc name=output-format,
			doc description={default: none, initial: \texttt{pdf}},
		},
		{
			% doc new=2025-01-01,
			% doc updated=2025-01-01,
			doc keypath=config,
			doc name=output_format,
			doc description={default: none, initial: \texttt{pdf}},
		},
	}
	Set output format.
	Possible values are
	\begin{multicols}{3}
		\begin{itemize}
			\item \texttt{pdf}
			\item \texttt{dvi}
			\item[]
		\end{itemize}
	\end{multicols}
\end{docKeys}
\begin{docKeys}[
		doc parameter={=\meta{check\_driver}},
	]{
		{
			% doc new=2025-01-01,
			% doc updated=2025-01-01,
			doc keypath=cli,
			doc name=check-driver,
			doc description={default: none, initial: none},
		},
		{
			% doc new=2025-01-01,
			% doc updated=2025-01-01,
			doc keypath=config,
			doc name=check_driver,
			doc description={default: none, initial: none},
		},
	}
	Check that the correct driver file is loaded for certain packages.
	Is one of
	\begin{multicols}{3}
		\begin{itemize}
			\item \texttt{dvipdfmx}
			\item \texttt{dvips}
			\item \texttt{dvisvgm}
		\end{itemize}
	\end{multicols}
	Can only be used with \refKey{/cli/output-format}\texttt{=dvi}.
\end{docKeys}

\subparagraph{Compilation}
\begin{docKeys}[
	]{
		{
			% doc new=2025-01-01,
			% doc updated=2025-01-01,
			doc keypath=cli,
			doc name=start-with-draft,
			doc description={enables when passed, initial: \texttt{false}},
		},
		{
			% doc new=2025-01-01,
			% doc updated=2025-01-01,
			doc keypath=cli,
			doc name=no-start-with-draft,
			doc description={disables when passed, initial: \texttt{false}},
		},
		{
			% doc new=2025-01-01,
			% doc updated=2025-01-01,
			doc keypath=config,
			doc name=start_with_draft,
			doc description={initial: \texttt{false}},
		},
	}
	\Todo{}
\end{docKeys}
\begin{docKeys}[
	]{
		{
			% doc new=2025-01-01,
			% doc updated=2025-01-01,
			doc keypath=cli,
			doc name=skip-first,
			doc description={enables when passed, initial: \texttt{false}},
		},
		{
			% doc new=2025-01-01,
			% doc updated=2025-01-01,
			doc keypath=cli,
			doc name=no-skip-first,
			doc description={disables when passed, initial: \texttt{false}},
		},
		{
			% doc new=2025-01-01,
			% doc updated=2025-01-01,
			doc keypath=config,
			doc name=skip_first,
			doc description={initial: \texttt{false}},
		},
	}
	\Todo{}
\end{docKeys}
\begin{docKeys}[
		doc parameter={=\meta{max\_iterations}},
	]{
		{
			% doc new=2025-01-01,
			% doc updated=2025-01-01,
			doc keypath=cli,
			doc name=max-iterations,
			doc description={default: none, initial: 3},
		},
		{
			% doc new=2025-01-01,
			% doc updated=2025-01-01,
			doc keypath=config,
			doc name=max_iterations,
			doc description={default: none, initial: 3},
		},
	}
	Set maximum number of run, for resolving cross-references and etc.
\end{docKeys}
\begin{docKeys}[
	]{
		{
			% doc new=2025-01-01,
			% doc updated=2025-01-01,
			doc keypath=cli,
			doc name=fresh,
			doc description={enables when passed, initial: \texttt{false}},
		},
		{
			% doc new=2025-01-01,
			% doc updated=2025-01-01,
			doc keypath=cli,
			doc name=no-fresh,
			doc description={disables when passed, initial: \texttt{false}},
		},
		{
			% doc new=2025-01-01,
			% doc updated=2025-01-01,
			doc keypath=config,
			doc name=fresh,
			doc description={initial: \texttt{false}},
		},
	}
	Clean auxiliary files before run.
	Cannot be used in conjunction with \refKey{/cli/output-directory}.
\end{docKeys}
\begin{docKeys}[
		doc parameter={=\meta{includeonly}},
	]{
		{
			% doc new=2025-01-01,
			% doc updated=2025-01-01,
			doc keypath=cli,
			doc name=includeonly,
			doc description={default: none, initial: none},
		},
		{
			% doc new=2025-01-01,
			% doc updated=2025-01-01,
			doc keypath=config,
			doc name=includeonly,
			doc description={default: none, initial: none},
		},
	}
	Insert \cs{includeonly}\marg{\meta{includeonly}}.
\end{docKeys}
\begin{docKeys}[
		doc parameter={=\meta{make\_depends}},
	]{
		{
			% doc new=2025-01-01,
			% doc updated=2025-01-01,
			doc keypath=cli,
			doc name=make-depends,
			doc description={default: none, initial: none},
		},
		{
			% doc new=2025-01-01,
			% doc updated=2025-01-01,
			doc keypath=config,
			doc name=make_depends,
			doc description={default: none, initial: none},
		},
	}
	Write Makefile-style dependencies information to \meta{make\_depends}.
	\Todo{more docs on this (somewhere else)?}
\end{docKeys}

\subparagraph{Console output}
% console output
% | help | help | h | - |
% | color | color[=color] | - | always |
% | quiet | quiet[=quiet] | - | 1 |
% | verbose | verbose | V | - |

\subsubsection{Output directory}
% | change_directory | change-directory | - | - |
% | output_directory | output-directory=output_directory | - | - |
% | print_output_directory | print-output-directory | - | - |

\subsubsection{Watch Mode}
% | watch | watch[=watch] | - | auto |
% | watch_not_ext | watch-not-ext=watch_not_ext | - | - |
% | watch_not_path | watch-not-path=watch_not_path | - | - |
% | watch_only_ext | watch-only-ext=watch_only_ext | - | - |
% | watch_only_path | watch-only-path=watch_only_path | - | - |

\subsubsection{Auxiliary/External Programs}

\paragraph{\BibTeX\,/\,Biber}
% | biber | biber[=biber] | - | biber |
% | bibtex | bibtex[=bibtex] | - | bibtex |

\paragraph{Glossaries}
% | glossaries | glossaries[=glossaries] | - | makeindex:main.glo:main.gls:main.glg |
% | makeindex | makeindex[=makeindex] | - | makeindex |

\paragraph{Sage\TeX}
% | sagetex | sagetex[=sagetex] | - | sage |

\subsubsection{Special packages}
\paragraph{Shell escaping}
% \paragraph{Minted + epstopd}
% | package_support | package-support=package_support | - | - |

\paragraph{memoize}
% | memoize | memoize[=memoize] | - | perl |
% | memoize_opt | memoize_opt=memoize_opt | - | - |


% TODO rewrite from here
\subsubsection{Basic options}
\begin{description}
\item[\texttt{-o}, \texttt{--output=\metavar{FILE}}]
  Set output file name.
  Default: \texttt{\metavar{JOBNAME}.\metavar{FORMAT}}
\item[\texttt{-q}, \texttt{--quiet[=\metavar{LEVEL}]}]
  Supress some output of the issued TeX command.\\
  Level 0: show all\\
  Level 1: avoid over-/underfull boxes (to some extend)\\
  Level 2: only show output generated inside document environment\\
  Default: \texttt{1}

  It is recomended to disable quiet output eventually to really make sure no warning goes unnoticed.
\item[\texttt{--fresh}]
  Clean auxiliary files before run.
  Cannot be used in conjunction with \texttt{--output-directory}.
\item[\texttt{--max-iterations=\metavar{N}}]
  Set maximum number of run, for resolving cross-references and etc.
  Default: 3
\item[\texttt{--skip-first}]
  Skips the first iteration by using the aux files of previous runs if the
  output is still newer than all other files. If the aux files are not found,
  this obviously will have no effect.
\item[\texttt{--watch[=\metavar{ENGINE}]}]
  Watch input files for change.
  May need an external program to be available.
  See \autoref{sec:watch-mode} for details.
\item[\texttt{--watch-only-path=\metavar{PATH}}]
\item[\texttt{--watch-not-path=\metavar{PATH}}]
\item[\texttt{--watch-only-ext=\metavar{EXT}}]
\item[\texttt{--watch-not-ext=\metavar{EXT}}]
  Watching engines often have an upper limit of how many files can be watched at once
  (inotifywait for instance seems to only be able to watch at most 1024\footnote{\url{https://github.com/inotify-tools/inotify-tools/blob/210b019fb621d32fd6986b512508fc845f6c9fcb/src/common.cpp\#L18C20-L18C24}} files). Thus, these options provide means to filter the files that shall be watched so that this limit is not exceeded.

  Note: You can use all of these options more than once.
  They will always be processed in the order you specified them (meaning the last option will always take precedence)

  Note: No matter which of these options you use, the default is always to not watch a file.
  So by only using \texttt{--watch-not-path=./aux/} you will end up by not watching any path.
  You can of course change this by specifying \texttt{--watch-only-path=/} before.
\item[\texttt{--color[=\metavar{WHEN}]}]
  Colorize messages.
  \metavar{WHEN} is one of \texttt{always}, \texttt{auto}, or \texttt{never}.
  If \texttt{--color} option is omitted, \texttt{auto} is used.
  If \metavar{WHEN} is omitted, \texttt{always} is used.
\item[\texttt{--includeonly=\metavar{NAMEs}}]
  Insert \texttt{\texcmd{includeonly}\{\metavar{NAMEs}\}}.
\item[\texttt{--make-depends=\metavar{FILE}}]
  Write Makefile-style dependencies information to \metavar{FILE}.
\item[\texttt{--engine-executable=\metavar{COMMAND}}]
  The actual \TeX\ command to use.
\item[\texttt{--tex-option=\metavar{OPTION}}, \texttt{--tex-options=\metavar{OPTIONs}}]
  Pass extra options to \TeX.
\item[\texttt{--dvipdfmx-option=\metavar{OPTION}}, \texttt{--dvipdfmx-options=\metavar{OPTIONs}}]
  Pass extra options to \texttt{dvipdfmx}.
\item[\texttt{--[no-]change-directory}]
  Change to the output directory when run.
  May be useful with shell-escaping packages.
\item[\texttt{-h}, \texttt{--help}]
\item[\texttt{-v}, \texttt{--version}]
\item[\texttt{-V}, \texttt{--verbose}]
\item[\texttt{--print-output-directory}]
  Print the output directory and exit.
\item[\texttt{--package-support=PKG1[,PKG2,...,PKGn]}]
  Enable special support for shell-escaping packages.
  Currently supported packages are `\texttt{minted}` and `\texttt{epstopdf}`.
\item[\texttt{--check-driver=DRIVER}]
  Check that the correct driver file is loaded for certain packages.
  \metavar{DRIVER} is one of \texttt{dvipdfmx}, \texttt{dvips}, or \texttt{dvisvgm}.
  Can only be used with \texttt{--output-format=dvi}.
\end{description}

\subsubsection{Options for running auxiliary programs}
\begin{description}
\item[\texttt{--makeindex=\metavar{COMMAND}}]
  Run MakeIndex.
\item[\texttt{--bibtex=\metavar{COMMAND}}]
  Run \BibTeX.
\item[\texttt{--biber[=\metavar{COMMAND}]}]
  Run Biber. Default value for \metavar{COMMAND}: \texttt{biber}
\item[\texttt{--sagetex[=\metavar{COMMAND}]}]
  Run sagetex. Experimental.
\item[\texttt{--glossaries[=\metavar{OPTION\_STRING}]}]
  Setup a custom glossary. See \autoref{sec:glossaries} for the syntax of \metavar{OPTION\_STRING} and examples.
\item[\texttt{--memoize[=\metavar{python|perl|extract-command}]}]
  Enable the hook for running an extract script or for the memoize package. Set
  either to \texttt{python} or \texttt{perl} (default) to select which
  extraction script to run or directly specify the executable which shall be
  run. This will also set the key \texttt{no memo dir} as cluttealtex itself
  already provides the feature of avoiding to clutter the working directory. It
  also sets the key \texttt{extract=no} since cluttealtex performs the
  extraction on its own.
\item[\texttt{--memoize\_opts=\metavar{PACKAGE OPT}}]
  Pass additional options like \texttt{readonly} to the memoize package. Pass
  this option multiple times to add more than one option.
  Especially \texttt{readonly} might be useful since this allows to build with
  the current set of memoized pictures and then after finishing writing a new
  picture memoize the new picture as well (compile once without
  \texttt{readonly}).
\end{description}

\subsubsection{\TeX-compatible options}
\begin{description}
\item[\texttt{--[no-]shell-escape}]
\item[\texttt{--shell-restricted}]
\item[\texttt{--synctex=\metavar{NUMBER}}]
  Generate Sync\TeX\ file.
  Note that \texttt{.synctex.gz} is created alongside the final \texttt{.pdf}.
  See \autoref{sec:synctex} for details.
\item[\texttt{--[no-]file-line-error}]
  Default: Yes
\item[\texttt{--[no-]halt-on-error}]
  Default: Yes
\item[\texttt{--interaction=\metavar{STRING}}]
\item[\texttt{--jobname=\metavar{STRING}}]
\item[\texttt{--fmt=\metavar{FORMAT}}]
\item[\texttt{--output-directory=\metavar{DIR}}]
  Set output directory for \TeX\ engine.
  Auxiliary files are produced in this directory.
  Default: somewhere in the temporary directory.
\item[\texttt{--output-format=\metavar{FORMAT}}]
  Set output format.
  Possible values are \texttt{pdf} or \texttt{dvi}.
  Default: \texttt{pdf}
\end{description}

Long options, except \TeX-compatible ones, need two hyphens (e.g. \texttt{-synctex=1} is accepted, but not \texttt{--color}).
Combining multiple short options, like \texttt{-Ve pdflatex}, is not supported.

\subsection{Sync\TeX}\label{sec:synctex}
You can generate Sync\TeX\ data with \texttt{--synctex=1} option.

Although \CluttealTeX\ has \enquote{Don't clutter your working directory} as its motto, the \texttt{.synctex.gz} file is always produced alongside the PDF file.
This is because Sync\TeX\ cannot find its data file if it's not in the same directory as the PDF.

\subsection{Watch mode}\label{sec:watch-mode}
If \texttt{--watch} option is given, \CluttealTeX\ enters \emph{watch mode} after processing the document.

On Windows, a built-in filesystem watcher is implemented.

On other platforms, an auxiliary program \texttt{fswatch}\footnote{\url{http://emcrisostomo.github.io/fswatch/}} or \texttt{inotifywait} needs to be installed.
The auxiliary program will be detected automatically, but you can also select one specific tool via the paramter of the \texttt{--watch} option.

\subsection{MakeIndex and \BibTeX}
If you want to generate index or bibliography, using MakeIndex or \BibTeX, set \texttt{--makeindex}, \texttt{--bibtex}, or \texttt{--biber} option.
You need to explicitly specify the command name as an argument (e.g. \texttt{--makeindex=makeindex}, \texttt{--bibtex=bibtex}).

If you want to use Biber to process bibliography, the option to use is \texttt{--biber}, not \texttt{--bibtex=biber}.

\subsubsection{Glossaries}
\label{sec:glossaries}
For more complex setups of indices you can use the \texttt{glossaries} option.
Its parameter takes the form
\texttt{type:outputExt:inputExt:logExt:pathToCommand:commandArgs} (colons
need to be escaped with \texttt{\textbackslash{}:}).

You might obmit trailing arguments which you do not need.

\begin{tabularx}{\linewidth}{cX}
	\texttt{type} &
	type of the glossary, used to determine which tool to run (e.g. \texttt{makeindex})
	\newline
	(this or \texttt{pathToCommand} are mandatory)
	\\
	\texttt{outputExt} &
	extension of the output file of the tool being run. The actual output file will
	be the tex-file you passed to \CluttealTeX{} with the extension replaced
	with this argument.
	\newline
	Same procedure is applied to the other extensions you pass (optionally) here.
	\newline
	(mandatory)
	\\
	\texttt{inputExt} &
	file generated by \LaTeX which then is being read by the tool being run
	\newline
	(optional -- generated from \texttt{outputExt} by using \texttt{.XXs} as extension)
	\\
	\texttt{logExt} &
	log file of the tool being run
	(optional -- generated from \texttt{outputExt} by using \texttt{.XXl} as extension)
	\\
	\texttt{pathToCommand} &
	specify an exact path to the tool which shall be run to avoid needing to
	put it in your \texttt{PATH}
	\newline
	(this or \texttt{pathToCommand} are mandatory)
	\\
	\texttt{commandArgs} &
	additional argument for the tool being run
\end{tabularx}

Examples:
\begin{itemize}
	\item  \texttt{makeindex:main.acr:main.acn:main.alg}
		\newline
		default setup for
		acronyms with the glossaries package
	\item \texttt{makeindex:main.glo:main.gls:main.glg}
		\newline
		default setup for the
		normal glossary with the glossaries package
		\newline
		(default if no argument is givent to the \texttt{glossaries} option)
\end{itemize}

\subsection{For writing a large document}
When writing a large document with \LaTeX, you usually split the \TeX\ files with \texcmd{include} command.
When doing so, \texcmd{includeonly} can be used to eliminate processing time.
But writing \texcmd{includeonly} in the \TeX\ source file is somewhat inconvenient.
After all, \texcmd{includeonly} is about \emph{how} to process the document, not about its content.

Therefore, \CluttealTeX\ provides an command-line option to use \texcmd{includeonly}.
See \autoref{sec:makefile-example} for example.

Tips: When using \texttt{includeonly}, avoid using \texttt{--makeindex} or \texttt{--biber}.

Another technique for eliminating time is, setting \texttt{--max-iterations=1}.
It stops \CluttealTeX\ from processing the document multiple times, which may take several extra minutes.

\subsection{Using Makefile}\label{sec:makefile-example}
You can create Makefile to avoid writing \CluttealTeX\ options each time.
Example:
\begin{verbatim}
main.pdf: main.tex chap1.tex chap2.tex
    cluttealtex -e lualatex -o $@ --makeindex=mendex $<

main-preview.pdf: main.tex chap1.tex chap2.tex
    cluttealtex -e lualatex -o $@ --makeindex=mendex --max-iterations=1 $<

chap1-preview.pdf: main.tex chap1.tex
    cluttealtex -e lualatex -o $@ --max-iterations=1 --includeonly=chap1 $<

chap2-preview.pdf: main.tex chap2.tex
    cluttealtex -e lualatex -o $@ --max-iterations=1 --includeonly=chap2 $<
\end{verbatim}

With \texttt{--make-depends} option, you can let \CluttealTeX\ infer sub-files and omit them from Makefile.
Example:

\begin{verbatim}
main.pdf: main.tex
    cluttealtex -e lualatex -o $@ --make-depends=main.pdf.dep $<

-include main.pdf.dep
\end{verbatim}

After initial \texttt{make} run, \texttt{main.pdf.dep} will contain something like this:
\begin{verbatim}
main.pdf: ... main.tex ... chap1.tex chap2.tex
\end{verbatim}

Note that \texttt{--make-depends} option is still experimental, and may not work well with other options like \texttt{--makeindex}.

\subsection{Default output directory}
The auxiliary files like \texttt{.aux} are generated somewhere in the temporary directory, by default.
The directory name depends on the following three parameters:
\begin{itemize}
\item The absolute path of the input file
\item \texttt{--jobname} option
\item \texttt{--engine} option
\end{itemize}
On the other hand, the following parameters doesn't affect the directory name:
\begin{itemize}
\item \texttt{--includeonly}
\item \texttt{--makeindex}, \texttt{--bibtex}, \texttt{--biber}, \texttt{--glossaries}
\end{itemize}

If you need to know the exact location of the automatically-generated output directory, you can invoke \CluttealTeX with \texttt{--print-output-directory}.
For example, \texttt{clean} target of your Makefile could be written as:
\begin{verbatim}
clean:
    -rm -rf $(shell cluttealtex -e pdflatex --print-output-directory main.tex)
\end{verbatim}

\CluttealTeX itself doesn't erase the auxiliary files, unless \texttt{--fresh} option is set.
Note that, the use of a temporary directory means, the auxiliary files may be cleared when the computer is rebooted.

\subsection{Support for \texpkg{minted} and \texpkg{epstopdf}}
In general, packages that execute external commands (shell-escape) don't work well with \texttt{-output-directory}.
Therefore, they don't work well with \CluttealTeX.

However, some packages provide a package option to let them know the location of \texttt{-output-directory}.
For example, \texpkg{minted} provides \texttt{outputdir}, and \texpkg{epstopdf} provides \texttt{outdir}.

\CluttealTeX\ can supply them the appropriate options, but only if it knows that the package is going to be used.
To let \CluttealTeX\ what packages are going to be used, use \texttt{--package-support} option.

For example, if you want to typeset a document that uses \texpkg{minted}, run the following:
\begin{verbatim}
cluttealtex -e pdflatex --shell-escape --package-support=minted document.tex
\end{verbatim}

\subsection{Check for driver file}

\CluttealTeX\ can check that the correct driver file is loaded when certain packages are loaded.
Currently, the list of supported packages are \texpkg{graphics}, \texpkg{color}, \texpkg{expl3}, \texpkg{hyperref}, and \texpkg{xy}.

The check is always done with PDF mode.
To check the driver with DVI mode, use \texttt{--check-driver} option.

\subsection{Configuration file}
When calling \CluttealTeX, it will search for a file named \verb|.cluttealtexrc.lua| in the current working directory. If it finds such a file, \CluttealTeX will load that lua file and use the tables defined there as configuration options. Options set via the cli will always have a higher priority and overwrite anything that was set via the configuration file.

Note: The configuration file must return the table used to configure \CluttealTeX

An example looks like this:
\begin{verbatim}
return {
	options = {
		file = "main.tex",
		output_directory = "tex-aux",
		change_directory = true,
		engine = "pdflatex",
		biber = true, -- uses the default value of biber
		glossaries = {type="makeindex", out="acr", inp="acn", log="alg"}, -- could also be an array of these tables to configure multiple glossaries
		max_iterations = "50",
		quiet = "0",
	},
	defaults = {
		watch = "inotify", -- changes the default value of this option. Only has an effect on cli arguments 
	}
}
\end{verbatim}

For now refer to \url{https://github.com/atticus-sullivan/cluttealtex/pull/22} for further details.
The docs will be rewritten soon and this option described in more detail.

\begin{landscape}
\section{Summary of all available Options} \label{sec:listOpt}
\begin{itemize}
	\item \emph{optname} refers to the name of the option when passing it via the config file (\texttt{.cluttealtexrc.lua})
	\item arguments containing \texttt{[=...]} all have a default value, so when passing this option you don't have to specify a value
\end{itemize}

\LTXtable{\linewidth}{args.tex}
\end{landscape}

\end{document}
