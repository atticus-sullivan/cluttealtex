\documentclass[a4paper]{report}
\usepackage[unicode]{hyperref}
\usepackage{amsmath}
\usepackage{tabularx}
\usepackage{xspace}
\usepackage{csquotes}
\newcommand\CluttealTeX{ClutTeal\TeX\xspace}
\providecommand\BibTeX{\textsc{Bib}\TeX\xspace}
\newcommand\texcmd[1]{\texttt{\textbackslash #1}}
\newcommand\texenv[1]{\texttt{#1}}
\newcommand\texpkg[1]{\texttt{#1}}
\newcommand\metavar[1]{\textnormal{\textsf{#1}}}

\title{\CluttealTeX manual\\(Version 0.8.0)}
\author{Lukas Heindl\thanks{thanks to ARATA Mizuki for writing cluttex of which \CluttealTeX is a fork}}
\date{2024-03-11}

\begin{document}
\maketitle
\tableofcontents

\chapter{About \CluttealTeX}
\CluttealTeX is an automation tool for \LaTeX\ document processing.
Basic features are,
\begin{itemize}
\item Does not clutter your working directory with ``extra'' files, like \texttt{.aux} or \texttt{.log}.
\item If multiple runs are required to generate correct document, do so.
\item Watch input files, and re-process documents if changes are detected\footnote{needs an external program if you are on a Unix system}.
\item Run MakeIndex, \BibTeX, Biber, if requested.
\item Produces a PDF, even if the engine (e.g.\ p\TeX) does not suport direct PDF generation.
  If you want a DVI file, use \texttt{--output-format=dvi} option.
\item No external dependencies, \CluttealTeX only uses texlua which should come with TeXLive
\end{itemize}

The unique feature of this program is that, auxiliary files such as \texttt{.aux} or \texttt{.toc} are created in an isolated location, so you will not be annoyed with these extra files.

% A competitor: \href{http://www.personal.psu.edu/jcc8/latexmk/}{Latexmk}

\chapter{How to use \CluttealTeX}
\section{Installation}
If you want to install \CluttealTeX manually, fetch an archive from
GitHub\footnote{\url{https://github.com/atticus-sullivan/cluttealtex}} (either
release for stable builds or from github actions for nightly build), extract
it, and copy \texttt{bin/cluttealtex} or \texttt{bin/cluttealtex.bat} to
somewhere in your \texttt{PATH}.

\subsection{Building}
You might also clone the repository and run \texttt{make install} (note the
required tools \texttt{fd}, \texttt{tl} (luarocks)) for installing to your
\texttt{TEXMFHOME} tree.

\section{Command-line usage}
Usage:
\begin{center}
  \texttt{cluttealtex -e \metavar{ENGINE} \metavar{OPTIONs} [--] \metavar{INPUT}.tex}
\end{center}

\subsection{Basic options}
\begin{description}
\item[\texttt{-e}, \texttt{--engine=\metavar{ENGINE}}]
  Set which \TeX\ engine/format to use.
  \metavar{ENGINE} is one of the following:
  \texttt{pdflatex}, \texttt{pdftex},
  \texttt{lualatex}, \texttt{luatex}, \texttt{luajittex},
  \texttt{xelatex}, \texttt{xetex},
  \texttt{latex}, \texttt{etex}, \texttt{tex},
  \texttt{platex}, \texttt{eptex}, \texttt{ptex},
  \texttt{uplatex}, \texttt{euptex}, or \texttt{uptex}.
  Required.
\item[\texttt{-o}, \texttt{--output=\metavar{FILE}}]
  Set output file name.
  Default: \texttt{\metavar{JOBNAME}.\metavar{FORMAT}}
\item[\texttt{--fresh}]
  Clean auxiliary files before run.
  Cannot be used in conjunction with \texttt{--output-directory}.
\item[\texttt{--max-iterations=\metavar{N}}]
  Set maximum number of run, for resolving cross-references and etc.
  Default: 3
\item[\texttt{--watch[=\metavar{ENGINE}]}]
  Watch input files for change.
  May need an external program to be available.
  See \autoref{sec:watch-mode} for details.
\item[\texttt{--watch-only-path=\metavar{PATH}}]
\item[\texttt{--watch-not-path=\metavar{PATH}}]
\item[\texttt{--watch-only-ext=\metavar{EXT}}]
\item[\texttt{--watch-not-ext=\metavar{EXT}}]
  Watching engines often have an upper limit of how many files can be watched at once
  (inotifywait for instance seems to only be able to watch at most 1024\footnote{\url{https://github.com/inotify-tools/inotify-tools/blob/210b019fb621d32fd6986b512508fc845f6c9fcb/src/common.cpp\#L18C20-L18C24}} files). Thus, these options provide means to filter the files that shall be watched so that this limit is not exceeded.

  Note: You can use all of these options more than once.
  They will always be processed in the order you specified them (meaning the last option will always take precedence)

  Note: No matter which of these options you use, the default is always to not watch a file.
  So by only using \texttt{--watch-not-path=./aux/} you will end up by not watching any path.
  You can of course change this by specifying \texttt{--watch-only-path=/} before.
\item[\texttt{--color[=\metavar{WHEN}]}]
  Colorize messages.
  \metavar{WHEN} is one of \texttt{always}, \texttt{auto}, or \texttt{never}.
  If \texttt{--color} option is omitted, \texttt{auto} is used.
  If \metavar{WHEN} is omitted, \texttt{always} is used.
\item[\texttt{--includeonly=\metavar{NAMEs}}]
  Insert \texttt{\texcmd{includeonly}\{\metavar{NAMEs}\}}.
\item[\texttt{--make-depends=\metavar{FILE}}]
  Write Makefile-style dependencies information to \metavar{FILE}.
\item[\texttt{--engine-executable=\metavar{COMMAND}}]
  The actual \TeX\ command to use.
\item[\texttt{--tex-option=\metavar{OPTION}}, \texttt{--tex-options=\metavar{OPTIONs}}]
  Pass extra options to \TeX.
\item[\texttt{--dvipdfmx-option=\metavar{OPTION}}, \texttt{--dvipdfmx-options=\metavar{OPTIONs}}]
  Pass extra options to \texttt{dvipdfmx}.
\item[\texttt{--[no-]change-directory}]
  Change to the output directory when run.
  May be useful with shell-escaping packages.
\item[\texttt{-h}, \texttt{--help}]
\item[\texttt{-v}, \texttt{--version}]
\item[\texttt{-V}, \texttt{--verbose}]
\item[\texttt{--print-output-directory}]
  Print the output directory and exit.
\item[\texttt{--package-support=PKG1[,PKG2,...,PKGn]}]
  Enable special support for shell-escaping packages.
  Currently supported packages are `\texttt{minted}` and `\texttt{epstopdf}`.
\item[\texttt{--check-driver=DRIVER}]
  Check that the correct driver file is loaded for certain packages.
  \metavar{DRIVER} is one of \texttt{dvipdfmx}, \texttt{dvips}, or \texttt{dvisvgm}.
  Can only be used with \texttt{--output-format=dvi}.
\end{description}

\subsection{Options for running auxiliary programs}
\begin{description}
\item[\texttt{--makeindex=\metavar{COMMAND}}]
  Run MakeIndex.
\item[\texttt{--bibtex=\metavar{COMMAND}}]
  Run \BibTeX.
\item[\texttt{--biber[=\metavar{COMMAND}]}]
  Run Biber. Default value for \metavar{COMMAND}: \texttt{biber}
\item[\texttt{--sagetex[=\metavar{COMMAND}]}]
  Run sagetex. Experimental.
\item[\texttt{--glossaries[=\metavar{OPTION\_STRING}]}]
  Setup a custom glossary. See \autoref{sec:glossaries} for the syntax of \metavar{OPTION\_STRING} and examples.
\item[\texttt{--memoize[=\metavar{python|perl|extract-command}]}]
  Enable the hook for running an extract script or for the memoize package. Set
  either to \texttt{python} or \texttt{perl} (default) to select which
  extraction script to run or directly specify the executable which shall be
  run. This will also set the key \texttt{no memo dir} as cluttealtex itself
  already provides the feature of avoiding to clutter the working directory. It
  also sets the key \texttt{extract=no} since cluttealtex performs the
  extraction on its own.
\item[\texttt{--memoize\_opts=\metavar{PACKAGE OPT}}]
  Pass additional options like \texttt{readonly} to the memoize package. Pass
  this option multiple times to add more than one option.
  Especially \texttt{readonly} might be useful since this allows to build with
  the current set of memoized pictures and then after finishing writing a new
  picture memoize the new picture as well (compile once without
  \texttt{readonly}).
\end{description}

\subsection{\TeX-compatible options}
\begin{description}
\item[\texttt{--[no-]shell-escape}]
\item[\texttt{--shell-restricted}]
\item[\texttt{--synctex=\metavar{NUMBER}}]
  Generate Sync\TeX\ file.
  Note that \texttt{.synctex.gz} is created alongside the final \texttt{.pdf}.
  See \autoref{sec:synctex} for details.
\item[\texttt{--[no-]file-line-error}]
  Default: Yes
\item[\texttt{--[no-]halt-on-error}]
  Default: Yes
\item[\texttt{--interaction=\metavar{STRING}}]
  \metavar{STRING} is one of \texttt{batchmode}, \texttt{nonstopmode}, \texttt{scrollmode}, or \texttt{errorstopmode}.
  Default: \texttt{nonstopmode}
\item[\texttt{--jobname=\metavar{STRING}}]
\item[\texttt{--fmt=\metavar{FORMAT}}]
\item[\texttt{--output-directory=\metavar{DIR}}]
  Set output directory for \TeX\ engine.
  Auxiliary files are produced in this directory.
  Default: somewhere in the temporary directory.
\item[\texttt{--output-format=\metavar{FORMAT}}]
  Set output format.
  Possible values are \texttt{pdf} or \texttt{dvi}.
  Default: \texttt{pdf}
\end{description}

Long options, except \TeX-compatible ones, need two hyphens (e.g. \texttt{-synctex=1} is accepted, but not \texttt{--color}).
Combining multiple short options, like \texttt{-Ve pdflatex}, is not supported.

\section{Sync\TeX}\label{sec:synctex}
You can generate Sync\TeX\ data with \texttt{--synctex=1} option.

Although \CluttealTeX\ has \enquote{Don't clutter your working directory} as its motto, the \texttt{.synctex.gz} file is always produced alongside the PDF file.
This is because Sync\TeX\ cannot find its data file if it's not in the same directory as the PDF.

\section{Watch mode}\label{sec:watch-mode}
If \texttt{--watch} option is given, \CluttealTeX\ enters \emph{watch mode} after processing the document.

On Windows, a built-in filesystem watcher is implemented.

On other platforms, an auxiliary program \texttt{fswatch}\footnote{\url{http://emcrisostomo.github.io/fswatch/}} or \texttt{inotifywait} needs to be installed.
The auxiliary program will be detected automatically, but you can also select one specific tool via the paramter of the \texttt{--watch} option.

\section{MakeIndex and \BibTeX}
If you want to generate index or bibliography, using MakeIndex or \BibTeX, set \texttt{--makeindex}, \texttt{--bibtex}, or \texttt{--biber} option.
You need to explicitly specify the command name as an argument (e.g. \texttt{--makeindex=makeindex}, \texttt{--bibtex=bibtex}).

If you want to use Biber to process bibliography, the option to use is \texttt{--biber}, not \texttt{--bibtex=biber}.

\subsection{Glossaries}
\label{sec:glossaries}
For more complex setups of indices you can use the \texttt{glossaries} option.
Its parameter takes the form
\texttt{type:outputExt:inputExt:logExt:pathToCommand:commandArgs} (colons
need to be escaped with \texttt{\textbackslash{}:}).

You might obmit trailing arguments which you do not need.

\begin{tabularx}{\linewidth}{cX}
	\texttt{type} &
	type of the glossary, used to determine which tool to run (e.g. \texttt{makeindex})
	\newline
	(this or \texttt{pathToCommand} are mandatory)
	\\
	\texttt{outputExt} &
	extension of the output file of the tool being run. The actual output file will
	be the tex-file you passed to \CluttealTeX{} with the extension replaced
	with this argument.
	\newline
	Same procedure is applied to the other extensions you pass (optionally) here.
	\newline
	(mandatory)
	\\
	\texttt{inputExt} &
	file generated by \LaTeX which then is being read by the tool being run
	\newline
	(optional -- generated from \texttt{outputExt} by using \texttt{.XXs} as extension)
	\\
	\texttt{logExt} &
	log file of the tool being run
	(optional -- generated from \texttt{outputExt} by using \texttt{.XXl} as extension)
	\\
	\texttt{pathToCommand} &
	specify an exact path to the tool which shall be run to avoid needing to
	put it in your \texttt{PATH}
	\newline
	(this or \texttt{pathToCommand} are mandatory)
	\\
	\texttt{commandArgs} &
	additional argument for the tool being run
\end{tabularx}

Examples:
\begin{itemize}
	\item  \texttt{makeindex:main.acr:main.acn:main.alg}
		\newline
		default setup for
		acronyms with the glossaries package
	\item \texttt{makeindex:main.glo:main.gls:main.glg}
		\newline
		default setup for the
		normal glossary with the glossaries package
		\newline
		(default if no argument is givent to the \texttt{glossaries} option)
\end{itemize}

\section{For writing a large document}
When writing a large document with \LaTeX, you usually split the \TeX\ files with \texcmd{include} command.
When doing so, \texcmd{includeonly} can be used to eliminate processing time.
But writing \texcmd{includeonly} in the \TeX\ source file is somewhat inconvenient.
After all, \texcmd{includeonly} is about \emph{how} to process the document, not about its content.

Therefore, \CluttealTeX\ provides an command-line option to use \texcmd{includeonly}.
See \autoref{sec:makefile-example} for example.

Tips: When using \texttt{includeonly}, avoid using \texttt{--makeindex} or \texttt{--biber}.

Another technique for eliminating time is, setting \texttt{--max-iterations=1}.
It stops \CluttealTeX\ from processing the document multiple times, which may take several extra minutes.

\section{Using Makefile}\label{sec:makefile-example}
You can create Makefile to avoid writing \CluttealTeX\ options each time.
Example:
\begin{verbatim}
main.pdf: main.tex chap1.tex chap2.tex
    cluttealtex -e lualatex -o $@ --makeindex=mendex $<

main-preview.pdf: main.tex chap1.tex chap2.tex
    cluttealtex -e lualatex -o $@ --makeindex=mendex --max-iterations=1 $<

chap1-preview.pdf: main.tex chap1.tex
    cluttealtex -e lualatex -o $@ --max-iterations=1 --includeonly=chap1 $<

chap2-preview.pdf: main.tex chap2.tex
    cluttealtex -e lualatex -o $@ --max-iterations=1 --includeonly=chap2 $<
\end{verbatim}

With \texttt{--make-depends} option, you can let \CluttealTeX\ infer sub-files and omit them from Makefile.
Example:

\begin{verbatim}
main.pdf: main.tex
    cluttealtex -e lualatex -o $@ --make-depends=main.pdf.dep $<

-include main.pdf.dep
\end{verbatim}

After initial \texttt{make} run, \texttt{main.pdf.dep} will contain something like this:
\begin{verbatim}
main.pdf: ... main.tex ... chap1.tex chap2.tex
\end{verbatim}

Note that \texttt{--make-depends} option is still experimental, and may not work well with other options like \texttt{--makeindex}.

\section{Default output directory}
The auxiliary files like \texttt{.aux} are generated somewhere in the temporary directory, by default.
The directory name depends on the following three parameters:
\begin{itemize}
\item The absolute path of the input file
\item \texttt{--jobname} option
\item \texttt{--engine} option
\end{itemize}
On the other hand, the following parameters doesn't affect the directory name:
\begin{itemize}
\item \texttt{--includeonly}
\item \texttt{--makeindex}, \texttt{--bibtex}, \texttt{--biber}, \texttt{--glossaries}
\end{itemize}

If you need to know the exact location of the automatically-generated output directory, you can invoke \CluttealTeX with \texttt{--print-output-directory}.
For example, \texttt{clean} target of your Makefile could be written as:
\begin{verbatim}
clean:
    -rm -rf $(shell cluttealtex -e pdflatex --print-output-directory main.tex)
\end{verbatim}

\CluttealTeX itself doesn't erase the auxiliary files, unless \texttt{--fresh} option is set.
Note that, the use of a temporary directory means, the auxiliary files may be cleared when the computer is rebooted.

\section{Support for \texpkg{minted} and \texpkg{epstopdf}}
In general, packages that execute external commands (shell-escape) don't work well with \texttt{-output-directory}.
Therefore, they don't work well with \CluttealTeX.

However, some packages provide a package option to let them know the location of \texttt{-output-directory}.
For example, \texpkg{minted} provides \texttt{outputdir}, and \texpkg{epstopdf} provides \texttt{outdir}.

\CluttealTeX\ can supply them the appropriate options, but only if it knows that the package is going to be used.
To let \CluttealTeX\ what packages are going to be used, use \texttt{--package-support} option.

For example, if you want to typeset a document that uses \texpkg{minted}, run the following:
\begin{verbatim}
cluttealtex -e pdflatex --shell-escape --package-support=minted document.tex
\end{verbatim}

\section{Check for driver file}

\CluttealTeX\ can check that the correct driver file is loaded when certain packages are loaded.
Currently, the list of supported packages are \texpkg{graphics}, \texpkg{color}, \texpkg{expl3}, \texpkg{hyperref}, and \texpkg{xy}.

The check is always done with PDF mode.
To check the driver with DVI mode, use \texttt{--check-driver} option.

\end{document}
